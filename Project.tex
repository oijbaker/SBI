% Options for packages loaded elsewhere
\PassOptionsToPackage{unicode}{hyperref}
\PassOptionsToPackage{hyphens}{url}
%
\documentclass[
]{article}
\usepackage{amsmath,amssymb}
\usepackage{iftex}
\ifPDFTeX
  \usepackage[T1]{fontenc}
  \usepackage[utf8]{inputenc}
  \usepackage{textcomp} % provide euro and other symbols
\else % if luatex or xetex
  \usepackage{unicode-math} % this also loads fontspec
  \defaultfontfeatures{Scale=MatchLowercase}
  \defaultfontfeatures[\rmfamily]{Ligatures=TeX,Scale=1}
\fi
\usepackage{lmodern}
\ifPDFTeX\else
  % xetex/luatex font selection
\fi
% Use upquote if available, for straight quotes in verbatim environments
\IfFileExists{upquote.sty}{\usepackage{upquote}}{}
\IfFileExists{microtype.sty}{% use microtype if available
  \usepackage[]{microtype}
  \UseMicrotypeSet[protrusion]{basicmath} % disable protrusion for tt fonts
}{}
\makeatletter
\@ifundefined{KOMAClassName}{% if non-KOMA class
  \IfFileExists{parskip.sty}{%
    \usepackage{parskip}
  }{% else
    \setlength{\parindent}{0pt}
    \setlength{\parskip}{6pt plus 2pt minus 1pt}}
}{% if KOMA class
  \KOMAoptions{parskip=half}}
\makeatother
\usepackage{xcolor}
\usepackage[margin=1in]{geometry}
\usepackage{color}
\usepackage{fancyvrb}
\newcommand{\VerbBar}{|}
\newcommand{\VERB}{\Verb[commandchars=\\\{\}]}
\DefineVerbatimEnvironment{Highlighting}{Verbatim}{commandchars=\\\{\}}
% Add ',fontsize=\small' for more characters per line
\usepackage{framed}
\definecolor{shadecolor}{RGB}{248,248,248}
\newenvironment{Shaded}{\begin{snugshade}}{\end{snugshade}}
\newcommand{\AlertTok}[1]{\textcolor[rgb]{0.94,0.16,0.16}{#1}}
\newcommand{\AnnotationTok}[1]{\textcolor[rgb]{0.56,0.35,0.01}{\textbf{\textit{#1}}}}
\newcommand{\AttributeTok}[1]{\textcolor[rgb]{0.13,0.29,0.53}{#1}}
\newcommand{\BaseNTok}[1]{\textcolor[rgb]{0.00,0.00,0.81}{#1}}
\newcommand{\BuiltInTok}[1]{#1}
\newcommand{\CharTok}[1]{\textcolor[rgb]{0.31,0.60,0.02}{#1}}
\newcommand{\CommentTok}[1]{\textcolor[rgb]{0.56,0.35,0.01}{\textit{#1}}}
\newcommand{\CommentVarTok}[1]{\textcolor[rgb]{0.56,0.35,0.01}{\textbf{\textit{#1}}}}
\newcommand{\ConstantTok}[1]{\textcolor[rgb]{0.56,0.35,0.01}{#1}}
\newcommand{\ControlFlowTok}[1]{\textcolor[rgb]{0.13,0.29,0.53}{\textbf{#1}}}
\newcommand{\DataTypeTok}[1]{\textcolor[rgb]{0.13,0.29,0.53}{#1}}
\newcommand{\DecValTok}[1]{\textcolor[rgb]{0.00,0.00,0.81}{#1}}
\newcommand{\DocumentationTok}[1]{\textcolor[rgb]{0.56,0.35,0.01}{\textbf{\textit{#1}}}}
\newcommand{\ErrorTok}[1]{\textcolor[rgb]{0.64,0.00,0.00}{\textbf{#1}}}
\newcommand{\ExtensionTok}[1]{#1}
\newcommand{\FloatTok}[1]{\textcolor[rgb]{0.00,0.00,0.81}{#1}}
\newcommand{\FunctionTok}[1]{\textcolor[rgb]{0.13,0.29,0.53}{\textbf{#1}}}
\newcommand{\ImportTok}[1]{#1}
\newcommand{\InformationTok}[1]{\textcolor[rgb]{0.56,0.35,0.01}{\textbf{\textit{#1}}}}
\newcommand{\KeywordTok}[1]{\textcolor[rgb]{0.13,0.29,0.53}{\textbf{#1}}}
\newcommand{\NormalTok}[1]{#1}
\newcommand{\OperatorTok}[1]{\textcolor[rgb]{0.81,0.36,0.00}{\textbf{#1}}}
\newcommand{\OtherTok}[1]{\textcolor[rgb]{0.56,0.35,0.01}{#1}}
\newcommand{\PreprocessorTok}[1]{\textcolor[rgb]{0.56,0.35,0.01}{\textit{#1}}}
\newcommand{\RegionMarkerTok}[1]{#1}
\newcommand{\SpecialCharTok}[1]{\textcolor[rgb]{0.81,0.36,0.00}{\textbf{#1}}}
\newcommand{\SpecialStringTok}[1]{\textcolor[rgb]{0.31,0.60,0.02}{#1}}
\newcommand{\StringTok}[1]{\textcolor[rgb]{0.31,0.60,0.02}{#1}}
\newcommand{\VariableTok}[1]{\textcolor[rgb]{0.00,0.00,0.00}{#1}}
\newcommand{\VerbatimStringTok}[1]{\textcolor[rgb]{0.31,0.60,0.02}{#1}}
\newcommand{\WarningTok}[1]{\textcolor[rgb]{0.56,0.35,0.01}{\textbf{\textit{#1}}}}
\usepackage{graphicx}
\makeatletter
\def\maxwidth{\ifdim\Gin@nat@width>\linewidth\linewidth\else\Gin@nat@width\fi}
\def\maxheight{\ifdim\Gin@nat@height>\textheight\textheight\else\Gin@nat@height\fi}
\makeatother
% Scale images if necessary, so that they will not overflow the page
% margins by default, and it is still possible to overwrite the defaults
% using explicit options in \includegraphics[width, height, ...]{}
\setkeys{Gin}{width=\maxwidth,height=\maxheight,keepaspectratio}
% Set default figure placement to htbp
\makeatletter
\def\fps@figure{htbp}
\makeatother
\setlength{\emergencystretch}{3em} % prevent overfull lines
\providecommand{\tightlist}{%
  \setlength{\itemsep}{0pt}\setlength{\parskip}{0pt}}
\setcounter{secnumdepth}{-\maxdimen} % remove section numbering
\ifLuaTeX
  \usepackage{selnolig}  % disable illegal ligatures
\fi
\IfFileExists{bookmark.sty}{\usepackage{bookmark}}{\usepackage{hyperref}}
\IfFileExists{xurl.sty}{\usepackage{xurl}}{} % add URL line breaks if available
\urlstyle{same}
\hypersetup{
  hidelinks,
  pdfcreator={LaTeX via pandoc}}

\author{}
\date{\vspace{-2.5em}}

\begin{document}

\hypertarget{sm2-sc2-project}{%
\subsubsection{SM2 / SC2 Project}\label{sm2-sc2-project}}

\hypertarget{using-abcsir-to-model-the-spread-of-influenza-in-a-boarding-school}{%
\section{Using ABC/SIR to Model the Spread of Influenza in a Boarding
School}\label{using-abcsir-to-model-the-spread-of-influenza-in-a-boarding-school}}

For this group project, we investigated the problem of intractable
likelihoods using Approximate Bayesian Computation. Such simulation
based methods are best applied to this case where we are working with a
model such that we are able to simulate results easily from it, yet have
no analytical form of the likelihood available. This is exactly the case
in epidemiology models, where we are unable to use methods such as
maximum likelihood estimation to estimate the infection parameters given
data.

\hypertarget{data-set}{%
\subsection{Data-set}\label{data-set}}

We begin by importing the dataset we chose for this project: the
\texttt{bsflu} dataset from the package \texttt{pomp}. This dataset
records a 1978 Influenza outbreak in a boy's boarding school.

\begin{Shaded}
\begin{Highlighting}[]
\FunctionTok{library}\NormalTok{(pomp)}
\FunctionTok{library}\NormalTok{(Rcpp)}
\FunctionTok{data}\NormalTok{(bsflu)}
\end{Highlighting}
\end{Shaded}

The dataset tallies infection information over a period of 14 days, in a
boarding school of 763 students.

\begin{Shaded}
\begin{Highlighting}[]
\FunctionTok{head}\NormalTok{(bsflu)}
\end{Highlighting}
\end{Shaded}

\begin{verbatim}
##         date   B  C day
## 1 1978-01-22   1  0   1
## 2 1978-01-23   6  0   2
## 3 1978-01-24  26  0   3
## 4 1978-01-25  73  1   4
## 5 1978-01-26 222  8   5
## 6 1978-01-27 293 16   6
\end{verbatim}

The column \texttt{B} contains the number of students who are bedridden
with the flu on a given day (i.e.~classes as `infected'). C contains the
number of students who are \emph{convalescent} (i.e.~not infected but
yet unable to return to class).

\hypertarget{model}{%
\subsection{Model}\label{model}}

In order to model disease data, we will use the well-studied SIR model.
This model models the number of people in three states: Susceptible,
Infected, and Recovered. The model is defined by the following system of
differential equations: \[
\begin{align*}
\frac{dS}{dt} &= -\beta S I \\
\frac{dI}{dt} &= \beta S I - \gamma I \\
\frac{dR}{dt} &= \gamma I
\end{align*}
\]

Where \(S\) is the proportion of susceptible individuals, \(I\) is the
proportion of infected individuals, and \(R\) is the proportion of
recovered individuals. \(\beta\) is the rate of infection, and
\(\gamma\) is the rate of recovery.

With such a definition, we can translate the column \texttt{B} in the
\texttt{bsflu} data directly to the variable \(I\) simply by dividing
\texttt{B} by the total number of students (\(N=763\)). Unfortunately,
the column \texttt{C} has no analogy in the model, as it acts as a
confusing `between recovery' state that cannot be grouped in with either
\(I\) or \(R\). Therefore going forward, we will primarily be using the
column \texttt{B} as the observed data in our SIR model estimate.

Approximate Bayesian computation is a simulation based approach, and
will require many individual computations. In the interest of speed
therefore, we will implement the SIR model in C++, then use RCPP to call
the C++ code from R.

\begin{Shaded}
\begin{Highlighting}[]
\FunctionTok{sourceCpp}\NormalTok{(}\AttributeTok{code =} \StringTok{"}
\StringTok{  \#include \textless{}Rcpp.h\textgreater{}}
\StringTok{  using namespace Rcpp;}

\StringTok{  // [[Rcpp::export]]}
\StringTok{  NumericVector SIR(NumericVector s, NumericVector i, NumericVector r, double s0, double i0, double r0, double beta, double gamma) \{}
\StringTok{    s[0] = s0;}
\StringTok{    i[0] = i0;}
\StringTok{    r[0] = r0;}

\StringTok{    for (int t = 1; t \textless{} s.size(); t++) \{}
\StringTok{      s[t] = s[t{-}1] {-} beta * s[t{-}1] * i[t{-}1];}
\StringTok{      i[t] = i[t{-}1] + beta * s[t{-}1] * i[t{-}1] {-} gamma * i[t{-}1];}
\StringTok{      r[t] = r[t{-}1] + gamma * i[t{-}1];}
\StringTok{    \}}

\StringTok{    return i;}
\StringTok{  \}}
\StringTok{"}\NormalTok{)}
\end{Highlighting}
\end{Shaded}

\hypertarget{abc-definition}{%
\subsection{ABC Definition}\label{abc-definition}}

Consider first the general case of intractable likelihood: having a
model \(f(.|\theta)\) with intractable likelihood \(l(\theta|.)\) and
parameter \(\theta\).

For ABC, we first repeatedly generate samples of \(\theta\) from a prior
distribution \(\theta \sim \pi(.)\). Then for each generated value of
\(\theta\), we input this into our effectively `black box' model to get
simulated data \(\tilde{y}(\theta) \sim f(.|\theta)\).

We then need to define some distance metric \(D\) between the observed
data \(y\) and simulated data \(\tilde{y}(\theta)\), only accepting the
proposed \(\theta\) if this distance falls below a defined tolerance
value \(\epsilon\).

Even a simple rejection sampling algorithm such as this can be shown to
produce samples \(\{\theta_1,...,\theta_M\}\) (\(M\) being the number of
accepted values) that are samples from the joint distribution:\[
\begin{align*}
    \pi_{\epsilon}(\theta,\tilde{y}|y) = \frac{\pi(\theta)f(\tilde{y}|\theta)\mathbb{I}(\tilde{y}\in A)}{\int_A\int_\Theta\pi(\theta)f(\tilde{y}|\theta) d \tilde{y} d \theta}
\end{align*}\] Where \(\mathbb{I}\) is the indicator function,
\(\Theta\) is the support of \(\theta\), and \(A\) is the acceptance
region defined by \(D\), \(y\), and \(\epsilon\). Then given a suitable
choice of tolerance value, this can produce an approximation to the
posterior distribution of \(\theta\) {[}1{]}.\[
\begin{align*}
    \pi_{\epsilon}(\theta|y) = \int_A\pi_{\epsilon}(\theta,\tilde{y}|y)d\tilde{y} \approx \pi(\theta|y)
\end{align*}\]

Clearly here much of the resulting estimate relies on our choice of
tolerance parameter \(\epsilon\) and distance metric \(D\), both of
which will be looked at later. Something else to consider is that in
practice the distance metric is applied to a \emph{summary statistic} of
the data rather than the raw data, in order to reduce dimensionality.
This can be anything from the mean \(\bar{y}\) and empirical quantiles,
to more complex statistics such as kernels or auxiliary parameters.
These methods will be looked at near the tail end of our investigation.

\hypertarget{abc-implementation}{%
\subsection{ABC Implementation}\label{abc-implementation}}

For the distance metric within ABC, we will need to compare the
simulated data to the observed data. However as noted in the Data-set
section, only the column B can be used. Hence we will compare the B
column of the dataset to the number of infected individuals I in our SIR
model.

\hypertarget{references}{%
\subsection{References}\label{references}}

{[}1{]} J.\emph{-M. Marin, P. Pudlo, C. P. Robert, and R. J. Ryder,
``Approximate bayesian computational methods,'' Stat Comput, vol.~22,
pp.~1167--1180, Nov.~2012.}

\end{document}
